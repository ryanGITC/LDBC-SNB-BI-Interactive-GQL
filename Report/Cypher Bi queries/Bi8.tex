\begin{listing}[!ht]
\begin{minted}
[
frame=lines,
framesep=2mm,
baselinestretch=1.2,
bgcolor=LightGray,
fontsize=\footnotesize,
linenos
]
{cypher}
/*Translation Bi8 query*/

CREATE Query Person_VIEW as (
FROM  Person
MATCH (tag:Tag {name: $tag})
// score
MATCH (tag)<-[interest:HAS_INTEREST]-(person:Person)
tag, collect(person) AS interestedPersons
WHERE EXIST{ 
    MATCH (tag)<-[:HAS_TAG]-(message:Message)-[:HAS_CREATOR]->(person:Person)
}
RETURN DISTINCT tag
       ,DISTINCT(interestedPersons + COLLECT (person) )AS persons
)

\end{minted}
\caption{Person view}
\label{8-1}
\end{listing}

The \ref{8-1} is composed with a WHERE EXIST CLAUSE.
In there the query starts with a left path traversal from
the message node towards the tag node, traversing through the path named \texttt{HAS\_TAG}. The other path \texttt{HAS\_CREATOR}
has a right traversal from the message node towards the person node. The inner clause is a condition, therefore the outer MATCH
clause first gets executed. In that clause, the person node is selected and from there on a traversal is done towards 
the tag node through the \texttt{HAS\_INTEREST} label.


\begin{listing}[!ht]
\begin{minted}
[
frame=lines,
framesep=2mm,
baselinestretch=1.2,
bgcolor=LightGray,
fontsize=\footnotesize,
linenos
]
{cypher}
FROM Person_VIEW
 MATCH (person)-[:KNOWS]-(friend)
// We need to use a redundant computation due to the lack of composable graph queries in the currently supported Cypher version.
// This might change in the future with new Cypher versions and GQL.
RETURN 
 tag,
  person,
  100 * size([(tag)<-[interest:HAS_INTEREST]-(person) | interest]) + size([(tag)<-[:HAS_TAG]-(message:Message)-[:HAS_CREATOR]->(person) WHERE message.creationDate > $date | message])
  AS score
  ,sum(score) AS friendScore

ORDER BY
  score + friendsScore DESC,
  person.id ASC
LIMIT 100

\end{minted}
\caption{Final query}
\label{8-2}
\end{listing}

The last query retrieves the results from the \texttt{person\_VIEW}. This however, might be redundant ,and could be constructed in one
query. But the current version of GQL described in the available resources does not mention composable graph queries.
As a result, a second query is constructed as  the final query. In it we have a traversal from the friend node
towards the person node. This traversal is a LUR and happens through the KNOWS path. As a final step the variables
described in table \ref{RetCypher8BITable} are returned.



\begin{table}[!ht]
\begin{tabular}{lllll}
\cline{1-2}
\multicolumn{1}{|l|}{\textbf{Variables}}                                                                                                                                                                                                                                       & \multicolumn{1}{l|}{\textbf{Meaning}}    &  &  &  \\ \cline{1-2}
\multicolumn{1}{|l|}{tag,}                                                                                                                                                                                                                                                     & \multicolumn{1}{l|}{the name of the tag} &  &  &  \\ \cline{1-2}
\multicolumn{1}{|l|}{100 * size({[}(tag)\textless{}-{[}interest:HAS\_INTEREST{]}-(person) | interest{]}) + size({[}(tag)\textless{}-{[}:HAS\_TAG{]}-(message:Message)-{[}:HAS\_CREATOR{]}-\textgreater{}(person) WHERE message.creationDate \textgreater \$date | message{]})} & \multicolumn{1}{l|}{}                    &  &  &  \\ \cline{1-2}
                                                                                                                                                                                                                                                                               &                                          &  &  &  \\
                                                                                                                                                                                                                                                                               &                                          &  &  & 
\end{tabular}
\caption{Returned variables and their meaning}
\label{RetCypher8BITable}
\end{table}